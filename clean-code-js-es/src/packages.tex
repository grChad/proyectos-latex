\usepackage[utf8]{inputenc} % Codificación de entrada
\usepackage[spanish]{babel} % Idioma del documento

\usepackage{geometry} % Configuración de márgenes
\usepackage{titlesec} % Personalización de títulos
\usepackage{lipsum} % Para generar texto de relleno, puedes eliminarlo
\usepackage{parskip} % Arreglo de la tabulación en el documento
\usepackage{lmodern} % Latin moderno
\usepackage{xcolor, soul} % Para poder usar colores
\usepackage{graphicx} % Imagenes
\usepackage{listings} % Para usar el paquete listings y sintaxis de códigos

% Configuración de márgenes
\geometry{
	left=2.0cm,
	right=1.5cm,
	top=1.5cm,
	bottom=1.5cm
}

% Para que el texto sea mas bonito
\renewcommand*{\familydefault}{\sfdefault}

% Colores
\definecolor{blue_lua}{RGB}{0, 153, 221} % Color personalizado para el Titulo
\definecolor{base}{RGB}{48, 52, 70}
\definecolor{pink}{RGB}{195, 66, 147}
\definecolor{comment}{RGB}{118, 126, 152}
\definecolor{flamingo}{RGB}{238, 190, 190}
\definecolor{maroon}{RGB}{204, 133, 136}
\definecolor{green}{RGB}{45, 194, 89}
\definecolor{red}{RGB}{234, 53, 56}

% Para insertar codigo en linea
\newcommand{\textblue}[1]{\textcolor{blue_lua}{#1}}
\newcommand{\textred}[1]{\textcolor{red}{#1}}
\newcommand{\lineCode}[1]{\colorbox{gray!10}{\textcolor{black!80}{\texttt{#1}}}}

% Configuración general para mostrar código
\lstdefinestyle{badstyle}{
	backgroundcolor=\color{red!5},
	keywordstyle=\color{pink}, % Estilo para las palabras clave
	commentstyle=\color{comment}\textit, % Estilo para los comentarios
	stringstyle=\color{green}, % Estilo para las cadenas de texto
	identifierstyle=\color{black!80},
	basicstyle=\color{maroon}\ttfamily,
	frame=none,
	breakatwhitespace=false,
	breaklines=true,
	keepspaces=true,
	showspaces=false,
	showstringspaces=false,
	showtabs=false,
	tabsize=3,
}

\lstdefinestyle{goodstyle}{
	backgroundcolor=\color{blue_lua!5},
	keywordstyle=\color{pink}, % Estilo para las palabras clave
	commentstyle=\color{comment}\textit, % Estilo para los comentarios
	stringstyle=\color{green}, % Estilo para las cadenas de texto
	identifierstyle=\color{black!80},
	basicstyle=\color{maroon}\ttfamily,
	frame=none,
	breakatwhitespace=false,
	breaklines=true,
	keepspaces=true,
	showspaces=false,
	showstringspaces=false,
	showtabs=false,
	tabsize=3,
}

\lstdefinestyle{mystyle}{
	backgroundcolor=\color{gray!7},
	keywordstyle=\color{pink}, % Estilo para las palabras clave
	commentstyle=\color{comment}\textit, % Estilo para los comentarios
	stringstyle=\color{green}, % Estilo para las cadenas de texto
	identifierstyle=\color{black!80},
	basicstyle=\color{maroon}\ttfamily,
	frame=none,
	breakatwhitespace=false,
	breaklines=true,
	keepspaces=true,
	showspaces=false,
	showstringspaces=false,
	showtabs=false,
	tabsize=3,
}

\lstdefinelanguage{TypeScript}{
	keywords={break, case, catch, class, const, continue, debugger, default, delete, do, else, enum, export, extends, false, finally, for, function, if, import, in, instanceof, new, null, return, super, switch, this, throw, true, try, typeof, var, void, while, with},
	morecomment=[l]{//},
	morecomment=[s]{/*}{*/},
	morestring=[b]",
	morestring=[b]'
}
