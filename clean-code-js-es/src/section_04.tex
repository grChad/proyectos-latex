\section{Objetos y estructuras de data}

\subsection*{Utiliza getters y setters}

Utilizar los getters y setters para acceder data dentro de los objetos puede ser mejor que simplemente buscar una propiedad. "Por qué?" Bueno, aquí te dejo con una lista desorganizada de las razones:

\begin{itemize}
  \item Cuando quieres hacer algo más allá de acceder una propiedad de objeto, no tienes qué buscar todos los accesorios en tu programa.
  \item Hace que implementar validación sea más fácil cuando construyes un \lineCode{set}.
  \item Encapsula la representación internal.
  \item Facilita la incorporación de apuntar errores de acceder y crear.
  \item Puedes cargar de manera vaga las propiedades del objeto, digamos de un servidor por ejemplo.
\end{itemize}

Mal Hecho:
\begin{lstlisting}[language=TypeScript, style=badstyle]
 function makeBankAccount() {
   // ...

   return {
     balance: 0,
     // ...
   };
 }

 const account = makeBankAccount();
 account.balance = 100;
\end{lstlisting}
\vspace{0.5cm} % space in line

Bien Hecho:
\begin{lstlisting}[language=TypeScript, style=goodstyle]
 function makeBankAccount() {
   let balance = 0;

   // a "getter", made public via the returned object below
   function getBalance() {
     return balance;
   }

   // a "setter", made public via the returned object below
   function setBalance(amount) {
     // ... validate before updating the balance
     balance = amount;
   }

   return {
     // ...
     getBalance,
     setBalance,
   };
 }

 const account = makeBankAccount();
 account.setBalance(100);
\end{lstlisting}
\newpage

\subsection*{Haz que los objetos tengan miembros privados}

Esto se puede lograr con \lineCode{closures} (con ES5 y antes).

Mal Hecho:
\begin{lstlisting}[language=TypeScript, style=badstyle]
 const Employee = function(name) {
   this.name = name;
 };

 Employee.prototype.getName = function getName() {
   return this.name;
 };

 const employee = new Employee('John Doe');
 console.log(`Employee name: ${employee.getName()}`); // Employee name: John Doe
 delete employee.name;
 console.log(`Employee name: ${employee.getName()}`); // Employee name: undefined
\end{lstlisting}
\vspace{0.5cm} % space in line

Bien Hecho:
\begin{lstlisting}[language=TypeScript, style=goodstyle]
 function makeEmployee(name) {
   return {
     getName() {
       return name;
     },
   };
 }

 const employee = makeEmployee('John Doe');
 console.log(`Employee name: ${employee.getName()}`); // Employee name: John Doe
 delete employee.name;
 console.log(`Employee name: ${employee.getName()}`); // Employee name: John Doe
\end{lstlisting}
