\section{Clases}

\subsection*{Prefiere ES2015/ES6 clases en vez de funciones normales de ES5}

Es muy difícil para obtener una herencia legible de las clases, las construcción y las definiciones de los métodos para las clases de ES5. Si necesitas la herencia (y puede que no la vayas a necesitar), entonces prefiere a las clases de ES2015/ES6. Sin embargo, prefiere funciones pequeñas hasta que necesites objetos más grandes y complejos.

Mal Hecho:

\begin{lstlisting}[language=TypeScript, style=badstyle]
 const Animal = function(age) {
   if (!(this instanceof Animal)) {
     throw new Error('Instantiate Animal with `new`');
   }

   this.age = age;
 };

 Animal.prototype.move = function move() {};

 const Mammal = function(age, furColor) {
   if (!(this instanceof Mammal)) {
     throw new Error('Instantiate Mammal with `new`');
   }

   Animal.call(this, age);
   this.furColor = furColor;
 };

 Mammal.prototype = Object.create(Animal.prototype);
 Mammal.prototype.constructor = Mammal;
 Mammal.prototype.liveBirth = function liveBirth() {};

 const Human = function(age, furColor, languageSpoken) {
   if (!(this instanceof Human)) {
     throw new Error('Instantiate Human with `new`');
   }

   Mammal.call(this, age, furColor);
   this.languageSpoken = languageSpoken;
 };

 Human.prototype = Object.create(Mammal.prototype);
 Human.prototype.constructor = Human;
 Human.prototype.speak = function speak() {};
\end{lstlisting}
\newpage

Bien Hecho:
\begin{lstlisting}[language=TypeScript, style=goodstyle]
 class Animal {
   constructor(age) {
     this.age = age;
   }

   move() { /* ... */ }
 }

 class Mammal extends Animal {
   constructor(age, furColor) {
     super(age);
     this.furColor = furColor;
   }

   liveBirth() { /* ... */ }
 }

 class Human extends Mammal {
   constructor(age, furColor, languageSpoken) {
     super(age, furColor);
     this.languageSpoken = languageSpoken;
   }

   speak() { /* ... */ }
 }
\end{lstlisting}
