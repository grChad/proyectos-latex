\section{Variables}

\subsection*{Utiliza nombres significativos y pronunciables para las variables}

Mal Hecho:
\begin{lstlisting}[language=TypeScript, style=badstyle]
 const yyyymmdstr = moment().format('YYYY/MM/DD');
\end{lstlisting}
\vspace{0.5cm} % space

Bien Hecho:
\begin{lstlisting}[language=TypeScript, style=goodstyle]
 const fechaActual = moment().format('YYYY/MM/DD');
\end{lstlisting}

\subsection*{Utiliza el vocabulario igual para las variables del mismo tipo}

Mal Hecho:
\begin{lstlisting}[language=TypeScript, style=badstyle]
 conseguirInfoUsuario();
 conseguirDataDelCliente();
 conseguirRecordDelCliente();
\end{lstlisting}
\vspace{0.5cm} % space

Bien Hecho:
\begin{lstlisting}[language=TypeScript, style=goodstyle]
 conseguirUsuario();
\end{lstlisting}

\subsection*{Utiliza nombres buscables}

Nosotros leemos mucho más código que jamás escribiremos. Es importante que el código que escribimos sea legible y buscable. Cuando faltamos nombrar a las variables de manera buscable y legible, acabamos confundiendo a nuestros lectores. Echa un vistazo a las herramientas para ayudarte: \textblue{\underline{buddy.js}} y \textblue{\underline{ESLint}}.

Mal Hecho:
\begin{lstlisting}[language=TypeScript, style=badstyle]
 // Para que rayos sirve 86400000?
 setTimeout(hastaLaInfinidadYMasAlla, 86400000);
\end{lstlisting}
\vspace{0.5cm} % space

Bien Hecho:
\begin{lstlisting}[language=TypeScript, style=goodstyle]
 // Declaralos como variables globales de 'const'.
 const MILISEGUNDOS_EN_UN_DIA = 8640000;

 setTimeout(hastaLaInfinidadYMasAlla, MILISEGUNDOS_EN_UN_DIA);
\end{lstlisting}

\newpage

\subsection*{Utiliza variables explicativas}

Mal Hecho:
\begin{lstlisting}[language=TypeScript, style=badstyle]
 const direccion = 'One Infinite Loop, Cupertino 95014';
 const codigoPostalRegex = /^[^,\\]+[,\\\s]+(.+?)\s*(\d{5})?$/;
 saveCityZipCode(direccion.match(codigoPostalRegex)[1], direccion.match(codigoPostalRegex)[2]);
\end{lstlisting}
\vspace{0.5cm} % space

Bien Hecho:
\begin{lstlisting}[language=TypeScript, style=goodstyle]
 const direccion = 'One Infinite Loop, Cupertino 95014';
 const codigoPostalRegex = /^[^,\\]+[,\\\s]+(.+?)\s*(\d{5})?$/;
 const [, ciudad, codigoPostal] = direccion.match(codigoPostalRegex) || [];
 guardarcodigoPostal(ciudad, codigoPostal);
\end{lstlisting}

\subsection*{Evitar el mapeo mental}

El explícito es mejor que el implícito.

Mal Hecho:
\begin{lstlisting}[language=TypeScript, style=badstyle]
 const ubicaciones = ['Austin', 'New York', 'San Francisco'];
 ubicaciones.forEach((u) => {
   hazUnaCosa();
   hasMasCosas()
   // ...
   // ...
   // ...
   // Espera, para que existe la 'u'?
   ejecuta(u);
 });
\end{lstlisting}
\vspace{0.5cm} % space

Bien Hecho:
\begin{lstlisting}[language=TypeScript, style=goodstyle]
 const ubicaciones = ['Austin', 'New York', 'San Francisco'];
 ubicaciones.forEach((ubicacion) => {
   hazUnaCosa();
   hazMasCosas()
   // ...
   // ...
   // ...
   ejecuta(ubicacion);
 });
\end{lstlisting}

\newpage

\subsection*{No incluyas contexto innecesario}

Si el nombre de tu clase/objeto te dice algo, no lo repitas de nuevo en el nombre de variable.

Mal Hecho:
\begin{lstlisting}[language=TypeScript, style=badstyle]
 const Coche = {
   cocheMarca: 'Honda',
   cocheModelo: 'Accord',
   cocheColor: 'Blue'
 };

 function pintarCoche(coche) {
   coche.cocheColor = 'Red';
 }
\end{lstlisting}
\vspace{0.5cm} % space

Bien Hecho:
\begin{lstlisting}[language=TypeScript, style=goodstyle]
 const Coche = {
   marca: 'Honda',
   modelo: 'Accord',
   color: 'Blue'
 };

 function pintarCoche(coche) {
   coche.color = 'Red';
 }
\end{lstlisting}

\subsection*{Utiliza argumentos predefinidos en vez de utilizar condicionales}

Los argumentos predefinidos muchas veces son más organizados que utilizar los condicionales. Se consciente que si tú los usas, tu función sólo tendrá valores para los argumentos de \lineCode{undefined}. Los demás valores de 'falso' como \lineCode{\textquotesingle\textquotesingle}, \lineCode{\textquotedblright\textquotedblright}, \lineCode{false}, \lineCode{null}, \lineCode{0} y \lineCode{NaN}, no se reemplazan con un valor predefinido.

Mal Hecho:
\begin{lstlisting}[language=TypeScript, style=badstyle]
 function crearEmpresa(nombre) {
   const nombreEmpresa = nombre || 'Tacos S.A';
   // ...
 }
\end{lstlisting}
\vspace{0.5cm} % space

Bien Hecho:
\begin{lstlisting}[language=TypeScript, style=goodstyle]
 function crearEmpresa(nombreEmpresa = 'Tacos S.A') {
   // ...
 }
\end{lstlisting}
