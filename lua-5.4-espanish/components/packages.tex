% NOTE: packates que se usaran en todo el proyecto
% paquetes para el idioma
\usepackage[T1]{fontenc}
\usepackage[spanish]{babel}
\defineshorthand{"-}{\babelhyphen{hard}} % para que los guiones o dashes no se fusionen

\usepackage[utf8]{inputenc}

\usepackage[top=25mm, left=25mm, bottom=25mm, right=18mm, headheight=25mm, b5paper]{geometry}

\usepackage{graphicx} % Imagenes
\usepackage{parskip} % Arreglo de la tabulación en el documento
\usepackage{xcolor, soul} % Para poder usar colores
\usepackage{titletoc} % Para personalizar la tabla de contenido
\usepackage{titlesec} % Para personalizar los títulos de los capítulos y las secciones
\usepackage{textcomp} % Agrega el paquete textcomp en el preámbulo
\usepackage{fancyhdr} % Para trabajar con el encabezado
\usepackage{amssymb} % Agrega el paquete amssymb en el preámbulo
\usepackage{listings} % Para usar el paquete listings y sintaxis de códigos
\usepackage{lipsum} % para generar texto aleatorio
\usepackage{booktabs} % Para usar tablas personalizadas.
\usepackage{lmodern} % Latin moderno
