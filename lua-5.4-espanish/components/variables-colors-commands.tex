% Colores
\definecolor{blue_lua}{RGB}{2,2,125} % Color personalizado para el Titulo
\definecolor{base}{RGB}{48, 52, 70}
\definecolor{pink}{RGB}{195, 66, 147}
\definecolor{comment}{RGB}{165, 173, 206}
\definecolor{flamingo}{RGB}{238, 190, 190}
\definecolor{maroon}{RGB}{204, 133, 136}
\definecolor{green}{RGB}{45, 194, 89}

% Item personalizado usando el cuadrado del paquete "amssymb"
\NewDocumentCommand{\listCustom}{m}{%
	\item[\hspace{5mm}$\scriptstyle\textcolor{black!70}{\blacksquare}$\hspace{3mm}{#1}]%
}

% Para insertar codigo en linea
\newcommand{\lineCode}[1]{\colorbox{gray!10}{\textcolor{black!80}{\texttt{#1}}}}
\newcommand{\texthigh}[1]{\textcolor{black!75}{\texttt{#1}}}

\newcommand{\keyword}[1]{\textcolor{pink}{\texttt{#1}}}

% para destacar referencias
\newcommand{\textRef}[1]{\textcolor{blue!80}{\texttt{#1}}}


\renewcommand*{\familydefault}{\sfdefault} % Para que el texto sea mas bonito
\renewcommand{\headrulewidth}{0pt} % el grosor del borde del encabezado


% --------------------[config fancyhdr]-------------------------------------
\fancyhf{} % cabezera y pie de pagina vacios
% izquierdo y pares
\fancyhead[LE]{\colorbox{gray!40}{\strut{\hspace{1mm}\thepage\hspace{1mm}}}\colorbox{black!70}{\strut{\hspace{1mm}\textcolor{white}{Capítulo \thechapter}\hspace{1mm}}}}

% derecho y impares
\fancyhead[RO]{\colorbox{black!70}{\strut{\hspace{1mm}\textcolor{white}{\nouppercase{\leftmark}\hspace{1mm}}}}\colorbox{gray!40}{\strut{\hspace{1mm}\thepage\hspace{1mm}}}}

\renewcommand{\chaptermark}[1]{\markboth{#1}{}} % para filtrar "Capítulo #" y quedarse con el nombre
% ---------------------end fancyhdr ----------------------------------------

% Configuración para el lenguaje Lua
\lstdefinelanguage{Lua}{
	keywords={and, break, do, else, elseif, end, false, for, function, if, in, local, nil, not, or, repeat, return, then, true, until, while},
	sensitive=true,
	comment=[l]{--},
	string=[b]",
	string=[b]',
	morestring=[b]",
	morestring=[b]',
}

% Configuración general para mostrar código
\lstdefinestyle{mystyle}{
	backgroundcolor=\color{gray!7},
	% keywordstyle=\color{pink}, % Estilo para las palabras clave
	keywordstyle=\color{pink}\textbf,
	commentstyle=\color{comment}, % Estilo para los comentarios
	stringstyle=\color{green}, % Estilo para las cadenas de texto
	% identifierstyle=\color{white},
	identifierstyle=\color{black!80}\textit,
	basicstyle=\color{maroon},
	frame=none,
	breakatwhitespace=false,
	breaklines=true,
	keepspaces=true,
	showspaces=false,
	showstringspaces=false,
	showtabs=false,
	tabsize=3,
	% xleftmargin=10pt,                        % Margen izquierdo
	% xrightmargin=10pt,                       % Margen derecho
}

% Definir el estilo personalizado para resaltar la sintaxis de una Terminal
\lstdefinestyle{terminalstyle}{
	basicstyle=\ttfamily\color{black!80},         % Estilo de fuente y color del texto
	% backgroundcolor=\color{base},            % Color de fondo
	frame=single,                             % Borde alrededor del contenido
	xleftmargin=10pt,                        % Margen izquierdo
	xrightmargin=10pt,                       % Margen derecho
	breaklines=true,                         % Romper líneas largas
	postbreak=\mbox{\textcolor{red}{$\hookrightarrow$}\space},  % Símbolo para líneas largas rotas
	showstringspaces=false,                  % No mostrar espacios en cadenas
}
