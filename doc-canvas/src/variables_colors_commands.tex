\renewcommand*{\familydefault}{\sfdefault} % Para que el texto sea mas bonito

% Colores
\definecolor{blue_lua}{RGB}{0, 153, 221} % Color personalizado para el Titulo
\definecolor{base}{RGB}{48, 52, 70}
\definecolor{pink}{RGB}{195, 66, 147}
\definecolor{comment}{RGB}{118, 126, 152}
\definecolor{flamingo}{RGB}{238, 190, 190}
\definecolor{maroon}{RGB}{204, 133, 136}
\definecolor{green}{RGB}{40, 180, 40}
\definecolor{yellow}{RGB}{250, 200, 20}
\definecolor{red}{RGB}{220, 30, 30}
\definecolor{orange}{RGB}{229, 83, 45}
\definecolor{blue}{RGB}{8, 120, 192}

% Para insertar código en linea
\newcommand{\textblue}[1]{\textcolor{blue_lua}{#1}}
\newcommand{\lineCode}[1]{\colorbox{gray!20}{\textcolor{black!90}{\texttt{#1}}}}
\newcommand{\code}[1]{\textcolor{pink!50!black}{\texttt{#1}}}
\newcommand{\textjs}[1]{\textcolor{yellow}{\textbf{#1}}}
\newcommand{\texthtml}[1]{\textcolor{orange}{\textbf{#1}}}
\newcommand{\textcss}[1]{\textcolor{blue}{\textbf{#1}}}
\newcommand{\taghtml}[1]{\textcolor{red}{\texttt{#1}}}

% Item personalizado usando el cuadrado del paquete "amssymb"
\NewDocumentCommand{\listCustom}{m}{%
	\item[\hspace{2.5mm}$\scriptstyle\textcolor{black!70}{\blacksquare}$\hspace{3mm}{#1}]%
}


% Configuración general para mostrar código
\lstdefinestyle{mystyle}{
	backgroundcolor=\color{gray!7},
	keywordstyle=\color{pink}, % Estilo para las palabras clave
	commentstyle=\color{comment}\textit, % Estilo para los comentarios
	stringstyle=\color{green}, % Estilo para las cadenas de texto
	identifierstyle=\color{black!80},
	basicstyle=\color{maroon}\ttfamily,
	frame=none,
	breakatwhitespace=false,
	breaklines=true,
	keepspaces=true,
	showspaces=false,
	showstringspaces=false,
	showtabs=false,
	tabsize=2,
}

\lstdefinestyle{mystyle2}{
	backgroundcolor=\color{gray!7},
	keywordstyle=\color{red}, % Estilo para las palabras clave
	commentstyle=\color{comment}\textit, % Estilo para los comentarios
	stringstyle=\color{green}, % Estilo para las cadenas de texto
	identifierstyle=\color{black!80},
	basicstyle=\color{teal}\ttfamily,
	frame=none,
	breakatwhitespace=false,
	breaklines=true,
	keepspaces=true,
	showspaces=false,
	showstringspaces=false,
	showtabs=false,
	tabsize=3,
}

\lstdefinelanguage{TypeScript}{
	keywords={break, case, catch, class, const, let, continue, debugger, default, delete, do, else, enum, export, extends, false, finally, for, function, if, import, in, instanceof, new, null, return, super, switch, this, throw, true, try, typeof, var, void, while, with},
	morecomment=[l]{//},
	morecomment=[s]{/*}{*/},
	morestring=[b]",
	morestring=[b]'
}

% Definición del lenguaje HTML
\lstdefinelanguage{HTML}{
	sensitive=true,
	keywords={html, head, canvas, title, meta, link, script, style, body, h1, h2, h3, p, div, span, a, img, ul, li, ol, table, tr, td, th },
	otherkeywords={<, >, /},
	morecomment=[s]{<!--}{-->},
	morestring=[b]',
	morestring=[b]",
}
