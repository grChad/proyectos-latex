% NOTE: packates que se usaran en todo el proyecto
% paquetes para el idioma
\documentclass[a4paper, 12pt]{article}

% Paquetes
\usepackage[utf8]{inputenc} % Codificación de entrada
\usepackage[spanish]{babel} % Idioma del documento
\usepackage[top=1.5cm, left=2.5cm, bottom=1.5cm, right=1.5cm]{geometry} % Configuración de márgenes

\usepackage{titlesec} % Personalización de títulos
\usepackage{lipsum} % Para generar texto de relleno, puedes eliminarlo
\usepackage{parskip} % Arreglo de la tabulación en el documento
\usepackage{lmodern} % Latin moderno
\usepackage{xcolor, soul} % Para poder usar colores
\usepackage{graphicx} % Imagenes
\usepackage{listings} % Para usar el paquete listings y sintaxis de códigos
\usepackage{tcolorbox} % Para agregar cuadros de notas muy bonitos
\usepackage{fontawesome} % Para usar los iconos de FontAwesome
\usepackage{upquote} % para poder usar las comillas dobles ``contenido'' = "contenido"
\usepackage{tikz} % para poder dibujar formas
\usepackage{pgfplots}   % Para gráficos más avanzados
\usepackage{amsmath}    % Para notación matemática (si es necesario)
\usepackage{amssymb} % Agrega el paquete amssymb en el preámbulo
