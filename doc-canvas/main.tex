% paquetes para el documento
% NOTE: packates que se usaran en todo el proyecto
% paquetes para el idioma
\usepackage[T1]{fontenc}
\usepackage[spanish]{babel}
\defineshorthand{"-}{\babelhyphen{hard}} % para que los guiones o dashes no se fusionen

\usepackage[utf8]{inputenc}

\usepackage[top=25mm, left=25mm, bottom=25mm, right=18mm, headheight=25mm, b5paper]{geometry}

\usepackage{graphicx} % Imagenes
\usepackage{parskip} % Arreglo de la tabulación en el documento
\usepackage{xcolor, soul} % Para poder usar colores
\usepackage{titletoc} % Para personalizar la tabla de contenido
\usepackage{titlesec} % Para personalizar los títulos de los capítulos y las secciones
\usepackage{textcomp} % Agrega el paquete textcomp en el preámbulo
\usepackage{fancyhdr} % Para trabajar con el encabezado
\usepackage{amssymb} % Agrega el paquete amssymb en el preámbulo
\usepackage{listings} % Para usar el paquete listings y sintaxis de códigos
\usepackage{lipsum} % para generar texto aleatorio
\usepackage{booktabs} % Para usar tablas personalizadas.
\usepackage{lmodern} % Latin moderno

\graphicspath{{./assets/}}

% Variables, colores y comandos
\renewcommand*{\familydefault}{\sfdefault} % Para que el texto sea mas bonito

% Colores
\definecolor{blue_lua}{RGB}{0, 153, 221} % Color personalizado para el Titulo
\definecolor{base}{RGB}{48, 52, 70}
\definecolor{pink}{RGB}{195, 66, 147}
\definecolor{comment}{RGB}{118, 126, 152}
\definecolor{flamingo}{RGB}{238, 190, 190}
\definecolor{maroon}{RGB}{204, 133, 136}
\definecolor{green}{RGB}{40, 180, 40}
\definecolor{yellow}{RGB}{250, 200, 20}
\definecolor{red}{RGB}{220, 30, 30}
\definecolor{orange}{RGB}{229, 83, 45}
\definecolor{blue}{RGB}{8, 120, 192}

% Para insertar código en linea
\newcommand{\textblue}[1]{\textcolor{blue_lua}{#1}}
\newcommand{\lineCode}[1]{\colorbox{gray!20}{\textcolor{black!90}{\texttt{#1}}}}
\newcommand{\code}[1]{\textcolor{pink!50!black}{\texttt{#1}}}
\newcommand{\textjs}[1]{\textcolor{yellow}{\textbf{#1}}}
\newcommand{\texthtml}[1]{\textcolor{orange}{\textbf{#1}}}
\newcommand{\textcss}[1]{\textcolor{blue}{\textbf{#1}}}
\newcommand{\taghtml}[1]{\textcolor{red}{\texttt{#1}}}

% Item personalizado usando el cuadrado del paquete "amssymb"
\NewDocumentCommand{\listCustom}{m}{%
	\item[\hspace{2.5mm}$\scriptstyle\textcolor{black!70}{\blacksquare}$\hspace{3mm}{#1}]%
}


% Configuración general para mostrar código
\lstdefinestyle{mystyle}{
	backgroundcolor=\color{gray!7},
	keywordstyle=\color{pink}, % Estilo para las palabras clave
	commentstyle=\color{comment}\textit, % Estilo para los comentarios
	stringstyle=\color{green}, % Estilo para las cadenas de texto
	identifierstyle=\color{black!80},
	basicstyle=\color{maroon}\ttfamily,
	frame=none,
	breakatwhitespace=false,
	breaklines=true,
	keepspaces=true,
	showspaces=false,
	showstringspaces=false,
	showtabs=false,
	tabsize=2,
}

\lstdefinestyle{mystyle2}{
	backgroundcolor=\color{gray!7},
	keywordstyle=\color{red}, % Estilo para las palabras clave
	commentstyle=\color{comment}\textit, % Estilo para los comentarios
	stringstyle=\color{green}, % Estilo para las cadenas de texto
	identifierstyle=\color{black!80},
	basicstyle=\color{teal}\ttfamily,
	frame=none,
	breakatwhitespace=false,
	breaklines=true,
	keepspaces=true,
	showspaces=false,
	showstringspaces=false,
	showtabs=false,
	tabsize=3,
}

\lstdefinelanguage{TypeScript}{
	keywords={break, case, catch, class, const, let, continue, debugger, default, delete, do, else, enum, export, extends, false, finally, for, function, if, import, in, instanceof, new, null, return, super, switch, this, throw, true, try, typeof, var, void, while, with},
	morecomment=[l]{//},
	morecomment=[s]{/*}{*/},
	morestring=[b]",
	morestring=[b]'
}

% Definición del lenguaje HTML
\lstdefinelanguage{HTML}{
	sensitive=true,
	keywords={html, head, canvas, title, meta, link, script, style, body, h1, h2, h3, p, div, span, a, img, ul, li, ol, table, tr, td, th },
	otherkeywords={<, >, /},
	morecomment=[s]{<!--}{-->},
	morestring=[b]',
	morestring=[b]",
}


% NOTE: ---------------------------------[INICIO]---------------------------------
\begin{document}

Por \textit{Jesús Gabriel Rivera}

En \faGithub \hspace{0.1cm} Github como \lineCode{\textit{grChad}}

Me he basado en información proporcionada por MDN (Mozilla Developer Network).

\hrulefill % Línea horizontal
\vspace{0.5cm} % space

\begin{center}
	\huge\textbf{\textblue{Tutorial de Canvas}}
	\vspace{0.5cm} % space

	\includegraphics[width=9cm]{logo-canvas} % Logo de Lua
	\vspace{0.5cm} % space
\end{center}

\taghtml{\textless canvas\textgreater} es un elemento \texthtml{HTML} el cual puede ser usado para dibujar gráficos usando \code{scripts} (normalmente \textjs{JavaScript}). Este puede, por ejemplo, ser usado para dibujar gráficos, realizar composición de fotos simples (y no tan simples) animaciones.

En este tutorial se describe cómo usar el elemento \taghtml{\textless canvas\textgreater} para dibujar gráficos en 2D, empezando con lo básico. Los ejemplos le proveerán mayor claridad a las ideas que pueda tener referentes al \code{canvas}, así como los códigos que necesita para crear su propio contenido.

\section{Introducción}

\begin{center}
  \includegraphics[width=11cm]{introduction_bad_code} % Logo de Lua
	\vspace{0.5cm} % space
\end{center}

Los principios de la ingeniería de software, del libro de Robert C. Martin \textblue{\textit{Clean Code}}, adaptado para JavasCript. Esta no es una guía de estilo, en cambio, es una guía para crear software que sea reutilizable, comprensible y que se pueda mejorar con el tiempo.

No hay que seguir tan estrictamente todos los principios en este libro, y vale la pena mencionar que hacia muchos de ellos habrá controversia en cuanto al consentimiento. Estas son reflexiones hechas después de muchos años de experiencia colectiva de los autores de \textit{Clean Code}.

Una cosa más: saber esto no te hará un mejor ingeniero inmediatamente, y tampoco trabajar con estas herramientas durante muchos años garantiza que nunca te equivocarás. Cualquier código empieza primero como un borrador, como arcilla mojada moldeándose en su forma final. Por último, arreglamos las imperfecciones cuando lo repasamos con nuestros compañeros de trabajo. No seas tan duro contigo mismo por los borradores iniciales que aún necesitan mejorar. ¡Trabaja más duro para mejorar el programa!
 % Primera sección
\section{Variables}

\subsection*{Utiliza nombres significativos y pronunciables para las variables}

Mal Hecho:
\begin{lstlisting}[language=TypeScript, style=badstyle]
 const yyyymmdstr = moment().format('YYYY/MM/DD');
\end{lstlisting}
\vspace{0.5cm} % space

Bien Hecho:
\begin{lstlisting}[language=TypeScript, style=goodstyle]
 const fechaActual = moment().format('YYYY/MM/DD');
\end{lstlisting}

\subsection*{Utiliza el vocabulario igual para las variables del mismo tipo}

Mal Hecho:
\begin{lstlisting}[language=TypeScript, style=badstyle]
 conseguirInfoUsuario();
 conseguirDataDelCliente();
 conseguirRecordDelCliente();
\end{lstlisting}
\vspace{0.5cm} % space

Bien Hecho:
\begin{lstlisting}[language=TypeScript, style=goodstyle]
 conseguirUsuario();
\end{lstlisting}

\subsection*{Utiliza nombres buscables}

Nosotros leemos mucho más código que jamás escribiremos. Es importante que el código que escribimos sea legible y buscable. Cuando faltamos nombrar a las variables de manera buscable y legible, acabamos confundiendo a nuestros lectores. Echa un vistazo a las herramientas para ayudarte: \textblue{\underline{buddy.js}} y \textblue{\underline{ESLint}}.

Mal Hecho:
\begin{lstlisting}[language=TypeScript, style=badstyle]
 // Para que rayos sirve 86400000?
 setTimeout(hastaLaInfinidadYMasAlla, 86400000);
\end{lstlisting}
\vspace{0.5cm} % space

Bien Hecho:
\begin{lstlisting}[language=TypeScript, style=goodstyle]
 // Declaralos como variables globales de 'const'.
 const MILISEGUNDOS_EN_UN_DIA = 8640000;

 setTimeout(hastaLaInfinidadYMasAlla, MILISEGUNDOS_EN_UN_DIA);
\end{lstlisting}

\newpage

\subsection*{Utiliza variables explicativas}

Mal Hecho:
\begin{lstlisting}[language=TypeScript, style=badstyle]
 const direccion = 'One Infinite Loop, Cupertino 95014';
 const codigoPostalRegex = /^[^,\\]+[,\\\s]+(.+?)\s*(\d{5})?$/;
 saveCityZipCode(direccion.match(codigoPostalRegex)[1], direccion.match(codigoPostalRegex)[2]);
\end{lstlisting}
\vspace{0.5cm} % space

Bien Hecho:
\begin{lstlisting}[language=TypeScript, style=goodstyle]
 const direccion = 'One Infinite Loop, Cupertino 95014';
 const codigoPostalRegex = /^[^,\\]+[,\\\s]+(.+?)\s*(\d{5})?$/;
 const [, ciudad, codigoPostal] = direccion.match(codigoPostalRegex) || [];
 guardarcodigoPostal(ciudad, codigoPostal);
\end{lstlisting}

\subsection*{Evitar el mapeo mental}

El explícito es mejor que el implícito.

Mal Hecho:
\begin{lstlisting}[language=TypeScript, style=badstyle]
 const ubicaciones = ['Austin', 'New York', 'San Francisco'];
 ubicaciones.forEach((u) => {
   hazUnaCosa();
   hasMasCosas()
   // ...
   // ...
   // ...
   // Espera, para que existe la 'u'?
   ejecuta(u);
 });
\end{lstlisting}
\vspace{0.5cm} % space

Bien Hecho:
\begin{lstlisting}[language=TypeScript, style=goodstyle]
 const ubicaciones = ['Austin', 'New York', 'San Francisco'];
 ubicaciones.forEach((ubicacion) => {
   hazUnaCosa();
   hazMasCosas()
   // ...
   // ...
   // ...
   ejecuta(ubicacion);
 });
\end{lstlisting}

\newpage

\subsection*{No incluyas contexto innecesario}

Si el nombre de tu clase/objeto te dice algo, no lo repitas de nuevo en el nombre de variable.

Mal Hecho:
\begin{lstlisting}[language=TypeScript, style=badstyle]
 const Coche = {
   cocheMarca: 'Honda',
   cocheModelo: 'Accord',
   cocheColor: 'Blue'
 };

 function pintarCoche(coche) {
   coche.cocheColor = 'Red';
 }
\end{lstlisting}
\vspace{0.5cm} % space

Bien Hecho:
\begin{lstlisting}[language=TypeScript, style=goodstyle]
 const Coche = {
   marca: 'Honda',
   modelo: 'Accord',
   color: 'Blue'
 };

 function pintarCoche(coche) {
   coche.color = 'Red';
 }
\end{lstlisting}

\subsection*{Utiliza argumentos predefinidos en vez de utilizar condicionales}

Los argumentos predefinidos muchas veces son más organizados que utilizar los condicionales. Se consciente que si tú los usas, tu función sólo tendrá valores para los argumentos de \lineCode{undefined}. Los demás valores de 'falso' como \lineCode{\textquotesingle\textquotesingle}, \lineCode{\textquotedblright\textquotedblright}, \lineCode{false}, \lineCode{null}, \lineCode{0} y \lineCode{NaN}, no se reemplazan con un valor predefinido.

Mal Hecho:
\begin{lstlisting}[language=TypeScript, style=badstyle]
 function crearEmpresa(nombre) {
   const nombreEmpresa = nombre || 'Tacos S.A';
   // ...
 }
\end{lstlisting}
\vspace{0.5cm} % space

Bien Hecho:
\begin{lstlisting}[language=TypeScript, style=goodstyle]
 function crearEmpresa(nombreEmpresa = 'Tacos S.A') {
   // ...
 }
\end{lstlisting}
 % Segunda sección

\end{document}
